AR(AR) is an emerging trend in the field of computing. AR’s growing popularity can partially be attributed to the rise of mobile phone use and a shift in AR applications no longer requiring expensive hardware and sophisticated equipment, such as head-mounted displays. Shared AR is a relatively new concept. It extends AR technologies, connecting multiple users together in order to create a synchronized, augmented space. This introduces a social, collaborative aspect to AR that lends itself to a range of use cases, from multiplayer gaming to training and education. Given the variety of exciting use cases for collaborative AR experiences, there has been significant advancements in the area. Tech giants such as Google and Apple are investing heavily in the sector, recently releasing software development kits for the creation of such AR experiences \cite{arcore} \cite{arkit}. In an attempt to evaluate the state of the field, the primary goal of this project is to investigate the various approaches and technologies currently used to implement multi-user AR experiences.

The field of architecture was selected as an ideal use case for demonstrating the value shared AR can provide. The architecture sector is an area requiring 3D visualizations and a need for common understanding and communication between client and architect. This project aims to highlight how these requirements can aptly be fulfilled through the use of shared AR, delivered via a cross-platform mobile application. An example use case could be an architect wishing to discuss architectural plans with a client. Each of the involved parties open up the mobile application and view the proposed architectural model in a shared augmented space. The model can be modified, re-positioned and scaled and the resulting changes will be viewed by all parties in the same shared AR space.  


\section{Context}
\subsection{AR}
Augmented (AR) is a 3D technology that enhances the user’s sensory perception of the real world with a contextual layer of information  \cite{azuma1997survey}. According to \cite{martin2015augmented}, the key to creating successful AR experiences lies in the smooth combination of the actual and virtual world. This merging of the real and virtual can be achieved using a variety of tools and methodologies. 

The primary method of implementing AR experiences is through the use of mobile devices and AR software development kits. Mobile devices offer an ideal hardware platform for AR applications. The portability, encouragement of high social interactivity, and independent operability \cite{hwang2012context} make mobile devices an attractive platform for AR applications. For this reason, it was decided that implementing shared AR in this project should be done via a mobile application. 

\subsection{Shared AR}
Shared AR or multi-user AR, allows multiple users to share a synchronized augmented space. Users see the same digital content simultaneously, creating a shared social experience that engages the users with the content, and with each other.  Each user can view the digital content from their own perspective – a user on one side of the room may see the front, while a user on the other side sees the rear. 


\section{Project Objectives}
The primary goal of this project is to demonstrate shared AR through the medium of a mobile application. The mobile application is intended to be used by architects and their clients as a means of promoting communication and collaboration between these parties.

\begin{itemize}
    \item Evaluate and establish the current tools available for creating cross-platform multi-user AR mobile applications.
    \item Implement shared AR.
    \item Build a full stack system for use in architecture, featuring shared AR technologies. 
\end{itemize}

\section{Summary} 
In this section a brief description of each chapter in this dissertation is provided. 

\subsection{Background}
This chapter describes the various technologies used during the development of this project. These technologies range from project management to database tools, to mobile application development frameworks. An evaluation of AR software development tool kits is also conducted in this chapter

\subsection{Methodology}
This chapter outlines the processes undertaken during the various stages of the project’s planning and development. It also describes the manner in which decisions regarding design and implementation were conducted. 

\subsection{System Design}
In this chapter, a detailed explanation of the overall system architecture is provided. The system’s key components are discussed and broken down into their underlying technologies. 

\subsection{System Evaluation}
This chapter evaluates the implemented system in terms of the initial project objectives. Final results are reviewed, along with a discussion on opportunities for improvement within the overall system.  

\subsection{Conclusion}
To conclude, a brief overview of the implemented system is provided. Key insights gained during the project's development are reflected upon and discussed.