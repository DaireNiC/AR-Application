This project sought to investigate and demonstrate the various methods and technologies currently used to create shared augmented reality experiences. As illustrated in section 5.1, while the end goals of the project were indeed achieved, the wealth of knowledge obtained during the process of its development can be considered the most valuable outcome of the project. 

During the research phase of this project, scholarly papers relating to shared augmented reality showed to be extremely scarce. Examining the tools for shared augmented reality creation is a novel concept and therefore an area in which research is particularly valued in the academic community. Additionally, exploring the state of the field in shared augmented reality, an area considered \say{bleeding edge}, led to gaining exposure to a vast range of new technologies. While it is certainly exciting to be investigating the latest of technologies, it is challenging at times due to the limited documentation.  As a result, progress can be easily hindered and often finding support to help resolve issues encountered can be difficult. Nevertheless, having to solve matters independently forces the developer to obtain a deep understanding of the technologies in use. 

Personally, this project led the developer to work with a wide selection of tools and processes never practised before, such as 3D modelling and encoding, native Android application development and mobile application development with Unity, to name only a few. As a result, a significant variety of new skills were amassed during the project’s development. 

On a final note, the experienced gained from developing a large-scale project solo was found to be immensely valuable. It served to highlight the importance of planning and self-discipline throughout a software development project. When there is genuine interest and excitement toward the project being developed, it can be tempting to dive straight into implementing the system. However, one of the key takeaways from this Applied Project and  Minor Dissertation is that a steady, consistent and incremental approach to software development is the most effective manner to ensure the successful completion of a project.
